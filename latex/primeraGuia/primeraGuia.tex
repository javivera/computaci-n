\documentclass[12pt]{article}

\usepackage[margin=1in]{geometry}
\usepackage{enumerate}
\usepackage{amsmath}
\usepackage{amssymb}
\usepackage{mathtools}
\usepackage{amsfonts}
\usepackage{amsthm}
\usepackage{graphicx}
\usepackage{fancyhdr}
\pagestyle{fancy}

	
\newcommand{\n}{\aleph_{0}}
\newcommand{\F}{\mathhbb{F}}
\newcommand{\Q}{\mathbb{Q}}
\newcommand{\C}{\mathbb{C}}
\newcommand{\R}{\mathbb{R}}
\newcommand{\K}{\mathbb{K}}
\newcommand{\E}{\mathbb{E}}
\newcommand{\I}{\mathbb{I}}
\newcommand{\Z}{\mathbb{Z}}
\newcommand{\N}{\mathbb{N}}
\newcommand{\Ra}{\Rightarrow}
\newcommand{\ra}{\rightarrow}
\newcommand{\ol}{\overline}
\newcommand{\norm}[1]{\left\lVert#1\right\rVert}
\newcommand{\open}{\mathrm{o}}


\theoremstyle{definition}
\newtheorem{definition}{Definición}[section]
\newtheorem*{remark}{Observación}
\newtheorem{theorem}{Teorema}
\newtheorem{lemm}{Lema}
\newtheorem{corollary}{Corolario}[theorem]
\newtheorem{lemma}[theorem]{Lema}
\newtheorem{prop}{Proposición}
\newtheorem{ej}{Ejercicio}


\fancyhead[R]{Compacidad}
\fancyhead[L]{Alumno Javier Vera}
\fancyhead[C]{Cálculo Avanzado}

\begin{document}
\begin{ej}
	Sean $p$ y $q$ variables proposicionales ¿Cuáles de las siguientes expresiones son fórmulas bien formadas?

	\begin{table}[h]
		\centering
		\begin{tabular}{llll}
			a) $(p \neg q)$ & d) $ \neg (p)$  & g) $(\neg p)$ \\
			b) $p \lor q \land True$& e) $(p \lor \neg p \land q)$ &  h) $(g \land False)$   \\
			c) $ (p \ra \neg p \ra q)$& f) $(True \land True \land True)$ &  i) $(p = q)$   \\
\end{tabular}
\end{table}

Respuestas:

\begin{table}[h]
		\centering
		\begin{tabular}{llll}
			a) NO  & d)  NO  & g) SI \\
			b) NO & e) NO  &  h) SI   \\
			c) SI & f) SI &  i) SI   \\
\end{tabular}
\end{table}


\end{ej}
\begin{ej}
	Respuestas
\end{ej}
\begin{table}[h]
		\centering
		\begin{tabular}{llll}
			a) Bien & d) Bien  \\
			b) Bien  & e) Mal    \\
			c) Mal & f) Mal    \\
\end{tabular}
\end{table}
\begin{ej}
	$3 + 7 = \pi - 8 $ es un tipo Bool dado que compara dos numeros , y luego comparamos un tipo bool con otro tipo bool
\end{ej}
\begin{ej}
	a) True b) True c) False d) True e) True f) True g) False
\end{ej}

\begin{ej}
	a) Tautología b) Contradicción c) Tautología d) Contingencia e) Tautologia f) Tautología g) Contingencia h) Tautología i)
	
\end{ej}
\begin{ej}
	
a) False mas fuerte que True $\quad$ b) $(p \land q)$ es mas fuerte que $(p\lor q)$ $\quad$c) Son igual de fuertes $\quad$ d) $(p \land q) $ es mas fuerte que p $\quad$ e) Son igual de fuertes $\quad$ f) p es mas fuerte que $(p \ra q)$ g) ninguna es mas fuerte que la otra. $\quad$ h) Ninguna es mas fuerte que la otra
\end{ej}

\begin{ej}
	a) Trabajemos el termino de la izquierda 

	$$ (1) \quad (\neg p \lor \neg q) \lor (p \land q) \ra (p \land q)$$ 

	$$ (2) \quad \neg ((\neg p \lor \neg q) \lor (p \land q)) \lor (p \land q) \text{ (No me acuerdo el nombre)}$$

	$$ (3) \quad (\neg (\neg p \lor \neg q) \land \neg (p \land q)) \lor (p \land q) \text{ (De Morgan)} $$

	$$ (4) \quad ((p \land q) \land (\neg p \lor \neg q)) \lor (p \land q) \text{ (Doble de Morgan)}$$ 

	$$ (5) \quad ((p \land q) \land \neg p) \lor ((p\land q) \land \neg q)) \lor (p \land q) \text{ (Distributiva)}$$

	$$ (6) \quad (False \lor False ) \lor (p \land q)$$

	$$ (7) \quad False \lor (p \land q)$$

	$$ (8) \quad p \land q$$


	b) 

	$$ (1) \quad (p \lor q) \land (p \lor r)$$

	$$ (2) \quad ((p \lor q) \land p) \lor ((p\lor q) \land r) \text{ (Distributiva)}$$

	$$ (3) \quad ((p \land p) \lor (p \land q)) \lor ((p \land r) \lor (q \land r)) \text{ (Distributiva)}$$

	$$ (4) \quad p \lor ((p \land r) \lor (q \land r)) $$

	$$ (5) \quad p \lor ((p \lor q) \land r)) \text{ (Inversa de Distributiva)}$$

	$$ (6) \quad (p \lor (p \lor q)) \land (p \land  r)  $$

	$$ (7) \quad (p \lor q) \land (p \land  r)  $$

	$$ (8) \quad p \lor (q \land  r)  $$

	$$ (9) \quad \neg p \ra (q \land  r)  $$

				Por ende eran equivalentes!

	c)
	$$ (1) \quad \neg(\neg p) \ra (\neg(\neg p \land \neg q))$$

	$$ (2) \quad  p \ra (p \lor q)$$ 

	$$ (3) \quad \neg  p \lor (p \lor q)$$

	$$ (4) \quad (\neg p \lor p ) \lor q \text{ (Asociatividad)}$$

	$$ (5) \quad True \lor q$$

	$$ (6) \quad True$$

	Entonces no son equivalentes!

	d)

	$$ (True \land p) \land (\neg p \lor False)) \ra \neg (\neg p \lor q) $$

	$$ (1) \quad ((True \land p) \land \neg p ) \lor ((True \land p) \land False) \ra \neg (\neg p \lor q) \text{ (Distributiva)}$$

	$$ (2) \quad (True \land (p \land \neg p)) \lor ((True \land False) \land p) \ra (p \land \neg q) \text { (Asoc y De Morgan)}$$


	$$ (3) \quad (True \land False ) \lor (False \land p) \ra (p \land \neg q)$$

	$$ (4) \quad (False \lor False )\ra (p \land \neg q) $$

	$$ (5) \quad \neg False \lor (p \land \neg q) $$

	$$ (6) \quad (True \lor p ) \land (True \lor \neg q)$$

	$$ (7) \quad True \land True$$

	$$ (8) \quad True $$

	Entonces no son equivalentes

	e) 

	$$ p \lor (\neg p \land q))$$

	$$ (1) \quad (p \lor \neg p) \land (p \lor q)$$

	$$ (2) \quad True \land (p \lor q)$$

	$$ (3) \quad (p \lor q)$$

	$$ (4) \quad \neg p \ra q$$

	f)

	$$ \neg (p \land (q \land s))$$

	$$ (1) \quad \neg ((p \land q) \land s)$$

	$$ (2) \quad \neg (p \land q) \lor \neg s$$

	$$ (3) \quad s \ra \neg (p\land q)  $$

	$$ (4) \quad s \ra (\neg p \lor \neg q)$$

	Entonces son equivalentes!

	g)

	$$ p \ra (q \land \neg (q \ra r))$$

	$$ (1) \quad \neg p \lor (q\land \neg(q \ra r))$$

	$$ (2) \quad (\neg p \lor q) \land (\neg p \lor \neg( q\ra r))$$

	Falta terminar. Será en otro momento

\end{ej}

\begin{ej}

	Muy amplio el ejercicio
	
\end{ej}
\begin{ej}
	a) 

	$$ f \ra (e \lor m) \land \neg (e \land m)$$

	$$ \neg f \ra \neg e \quad (f \lor \neg e) $$

	$$  f \land e \ra m \quad (\neg (f \land e) \lor m)$$

	b) Veamos la primera afirmación


	$$ (\neg f \lor ((e \lor m) \land \neg (e \land m) )\land (f \lor \neg e) \land (\neg (f \land e) \lor m) \ra \neg e$$

	$$ (\neg f \lor ((e \lor m) \land \neg (e \land m) )\land (f \lor \neg e) \land (\neg (f \land e) \lor m) \ra \neg e$$

	Bueno hay que seguir hasta llegar a un TRUE , pero no lo voy a hacer
\end{ej}

\begin{ej} Definamos algunas equivalencias

	$p \equiv$ todos conocen a Juan

	$q \equiv$ todos conocen a Camila

	$r \equiv$ todos conocen a Gonzalo

	Ahora sabemos

	$$p \ra q$$

	$$p \ra (q \ra r) $$

	Ahora sabemos que $p$ es verdadero

	Pero entonces $q$ es verdadero 

	Ademas p es verdadero entonces $(q\ra r)$ es verdadero

	Pero como sabiamos que $q$ era verdadero entonces $r$ es necesariamente verdadero

	Entonces deducimos $p \ra r$ por lo tanto si todos conocen a Juan , todos conocen a Gonzalo
\end{ej}

\begin{ej}
	Supongamos $p \equiv$ Haroldo Pelea y $q \equiv$ Haroldo vuelve lastimado.
	Ahora sabemos que es cierto $p \ra q$ (Haroldo pelea entonces vuelve lastimado).
	Teniendo esto sabemos que no es cierto que si Haroldo vuelve lastimado entonces peleó, se puede ver facilmente haciendo la tabla de $p \ra q$ , si $p$ es verdadero $q$ también lo és. Pero si $q$ es verdadero $p$ puede o no ser verdadero
\end{ej}


\begin{ej}
	a) T b) T c)T d) F e) Undef f) F g) T h) T i) T
\end{ej}

\begin{ej}
	Se hizo en la teórica la diferencia es que hay un caso con indefinido que es False , es el caso de $p \land q$ , con p False y q indefinido, si no fuese el $\land_{L} $ entonces evaluaria indefinido , pero con el $\land_{L}$ al tener un false en $p$ automaticamente la afirmación no puede sere verdadera y evalua a falsa
\end{ej}

\begin{ej}
	Sucede algo similar solo que en este caso $p$ es True , por ende $p \lor_{L}$ lo que sea automaticamente evalua a True
\end{ej}

\begin{ej}
	Devuelta algo similar solo que funciona cuando $p$ es verdadero $p \ra_{L} q$ automaticamente es Verdadero sin importar si $q$ está indefinido 
\end{ej}

\begin{ej}
	a) Indef b) True c) False d) True e) True f) True g) False

\end{ej}
    
\begin{ej}
Hay que tener imaginación

\end{ej}

\begin{ej}
a) Sale a ojo	

 b) 

 I) n = 1 y=1 z=1

 II) n = m = 1 z =0

 III ) no tiene libres

 IV) esta mal formada

 El resto no tienen variables libres
\end{ej}

\begin{ej} Veamos:

a) Versión corregida $pred$ $a()\{(\forall x : \N)(((0 \leq x < 10) \land P(x)) \ra Q(x) \}$	

b) Versión corregida $pred$ $a()\{\neg (\exists x : \Z)((0 \leq x < 10) \land P(x) \land Q(x))\} $
\end{ej}

\begin{ej} Solución:

	a) $aux$ $suc$ $(x : \Z): \Z = x +1$

	b) $aux$ $suma$ $(x,y : \R): \R = x +y$

	c) $aux$ $producto$ $(x,y : \R): \R = xy$

	d) $pred$ $esCuadrado$ $(x : \Z)\{x > 0 \land \sqrt{x} \in \Z \}$

	e) $pred$ $esPrimo$ $(x:\Z)\{(x > 0 ) \land (\forall x'\in \N_{0})(x'< x \ra x $ mod $ x' \neq 0)\}$

	f) $pred$ $sonCoprimos$ $(x,y:\Z)\{(\forall d \in \N_{> 1} )(x $ mod $  d = 0 \ra y $ mod $ d \neq 0)\}$

	g) $pred$ $divisoresGrandes$ $(x,y : \Z)\{(\forall r \in \Z)((x$ mod $r \land r \neq 1 )> y) \}$
	
	h) $pred$ $mayorPrimoQueDivide$ $(x,y : \Z)\{esPrimo(y) \land (x$ mod  $y = 0) \land (\forall j \in \Z)((esPrimo(j) \land x $ mod $ j=0) \ra j < y) \}$

	i) $pred$ $sonPrimoHermanos$ $(x,y : \Z)\{esPrimo(x) \land esPrimo(y) \land (\neg \exists j \in \N)(esPrimo(j) \land (x<j<y) \lor (y<j<x) )\}$


\end{ej}

\end{document}


